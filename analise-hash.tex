\chapter{Análise de Desempenho dos Métodos de Hash}

Neste capítulo, analisaremos o desempenho das principais estratégias de implementação de tabelas de dispersão, com ênfase nas operações de inserção e busca. 
Consideraremos tanto o \textbf{pior caso} quanto o \textbf{caso médio}, comparando os métodos de encadeamento e de endereçamento aberto sob diferentes condições de uso.

A análise será feita utilizando a \textbf{notação assintótica} $\Theta$, que descreve a ordem de crescimento do tempo de execução das operações em função do número de elementos armazenados. 
Quando dizemos que uma operação tem custo $\Theta(n)$, queremos dizer que o tempo de execução cresce proporcionalmente a $n$, tanto como limite superior quanto inferior. 
De forma análoga, $\Theta(1)$ indica que o tempo de execução permanece constante, independentemente do tamanho da entrada.

\begin{center}
\noindent\fbox{%
    \parbox{0.95\textwidth}{%
        \textbf{Hipótese da função de hash uniforme:} Para todas as análises de caso médio apresentadas neste capítulo, assumiremos que a função de hash utilizada distribui as chaves de forma \textbf{uniforme} e \textbf{aleatória} entre as posições da tabela. 
    }
}
\end{center}

Seja $n$ o número total de elementos armazenados na tabela e $m$ o tamanho da tabela (isto é, o número de posições disponíveis). 
O \textbf{fator de carga} é definido por:

\[
\alpha = \frac{n}{m}
\]

Esse fator expressa a média de elementos por posição da tabela. 
No caso de encadeamento, representa o comprimento médio das listas associadas a cada posição. 
No caso de endereçamento aberto, indica a proporção da tabela que está ocupada.

\section{Encadeamento}

No encadeamento, cada posição da tabela aponta para uma lista encadeada contendo os pares chave–valor cujas chaves foram mapeadas para aquela posição pela função de hash. Essa abordagem é simples, flexível e lida bem com colisões, desde que o fator de carga seja mantido sob controle.

\subsection*{Custo de busca}

A seguir, analisamos o custo da operação de busca, tanto no caso médio quanto no pior caso. 
Para essa análise, usamos a notação $\Theta$ para indicar a ordem exata de crescimento do tempo de execução em função do número de elementos armazenados.

Sob a hipótese de que a função de hash distribui as chaves uniformemente entre as $m$ posições da tabela, temos:

\begin{itemize}
  \item \textbf{Busca bem-sucedida:} a chave está presente e espera-se que esteja distribuída uniformemente. 
  Nesse caso, o custo médio é
  \[
  \Theta\left(1 + \frac{\alpha}{2}\right)
  \]
  pois percorremos, em média, metade da lista associada à posição. 
  Em particular, se garantimos que $n = O(m)$, ou seja, o número de elementos cresce no máximo linearmente com o tamanho da tabela, o custo médio torna-se
  \[
  \Theta(1)
  \]

  \item \textbf{Busca mal-sucedida:} a chave não está presente e percorremos toda a lista daquela posição para verificar sua ausência. 
  O custo médio é
  \[
  \Theta(\alpha)
  \]
\end{itemize}

No pior caso, todas as $n$ chaves colidem na mesma posição da tabela, formando uma única lista encadeada. 
A busca, nesse cenário, precisa percorrer a lista inteira:

\[
\Theta(n)
\]

\subsection{Custo de inserção}

A inserção em tabelas com encadeamento consiste em adicionar um novo par chave–valor na lista associada à posição calculada pela função de hash. 
Se a chave já estiver presente, o valor é atualizado; caso contrário, um novo nó é adicionado.

Assumindo novamente uma distribuição uniforme das chaves, espera-se que o comprimento médio de cada lista seja proporcional ao fator de carga~$\alpha$. 
Como a inserção é realizada no início da lista (sem necessidade de percorrê-la), o custo da operação é constante:

\[
\Theta(1)
\]

No pior caso, todas as chaves colidem na mesma posição da tabela, formando uma única lista encadeada de comprimento~$n$. 
Se for necessário verificar a existência prévia da chave antes de inserir (como em um dicionário), é preciso percorrer toda a lista. 
O custo, portanto, é:

\[
\Theta(n)
\]

\section{Endereçamento Aberto}

No endereçamento aberto, todos os elementos são armazenados diretamente na tabela. 
Quando ocorre uma colisão, a função de hash é ajustada de forma sistemática para sondar outras posições disponíveis. 
Neste modelo, não há listas auxiliares: todas as chaves ocupam posições dentro do vetor principal.

No \emph{linear probing}, em caso de colisão, procuramos sequencialmente pela próxima posição livre na tabela. 

\subsection*{Caso médio}

Assumindo que as chaves são distribuídas de forma uniforme e aleatória e que a tabela ainda não está muito cheia (isto é, $\alpha < 1$), temos os seguintes custos médios para busca no \emph{linear probing}, conforme a análise clássica de Knuth\footnote{Donald E. Knuth. \textit{The Art of Computer Programming}, Volume 3: \textit{Sorting and Searching}, 2nd ed., Addison-Wesley, 1998. Seção 6.4.}:

\begin{itemize}
  \item \textbf{Busca mal-sucedida:}
  \[
  \Theta\left( \frac{1}{2} \left(1 + \frac{1}{(1 - \alpha)^2} \right) \right)
  \]

  \item \textbf{Busca bem-sucedida:}
  \[
  \Theta\left( \frac{1}{2} \left(1 + \frac{1}{1 - \alpha} \right) \right)
  \]
\end{itemize}

Não desenvolveremos aqui a demonstração dessas expressões, mas elas ilustram como o custo médio de busca cresce rapidamente à medida que o fator de carga $\alpha$ se aproxima de 1. 
Em particular, o desempenho se mantém eficiente apenas enquanto $\alpha$ for significativamente menor do que 1.

\section{Comparação dos métodos}

As Tabelas a seguir resumem os custos médios e de pior caso das operações de busca e inserção nas principais estruturas discutidas. 
A lista ligada é apresentada como referência: embora tenha inserção eficiente no início da lista, sua busca é linear no número de elementos. 
O encadeamento oferece inserção rápida e desempenho de busca que depende do fator de carga~$\alpha = n/m$; quando esse fator é mantido constante, tanto a busca quanto a inserção apresentam custo constante no caso médio. 
No endereçamento aberto com sondagem linear, o custo médio das operações cresce rapidamente à medida que~$\alpha$ se aproxima de 1, conforme indicado pelas expressões assintóticas. 
Em todos os métodos, o pior caso pode chegar a~$\Theta(n)$, o que reforça a importância de uma boa função de hash e do controle rigoroso do fator de carga.

\begin{table}[h!]
\centering
\caption{Desempenho médio das operações}
\begin{tabular}{|l|c|c|}
\hline
\textbf{Estrutura} & \textbf{Busca (média)} & \textbf{Inserção (média)} \\
\hline
Lista ligada        & $\Theta(n)$ & $\Theta(1)$ \\
Encadeamento        & $\Theta(1 + \alpha)$ & $\Theta(1)$ \\
Sondagem linear     & $\Theta\left( \frac{1}{1 - \alpha} \right)$ & $\Theta\left( \frac{1}{1 - \alpha} \right)$ \\
\hline
\end{tabular}
\end{table}


\begin{table}[h!]
\centering
\caption{Desempenho no pior caso}
\begin{tabular}{|l|c|c|}
\hline
\textbf{Estrutura} & \textbf{Busca (pior)} & \textbf{Inserção (pior)} \\
\hline
Lista ligada        & $\Theta(n)$ & $\Theta(1)$ \\
Encadeamento        & $\Theta(n)$ & $\Theta(n)$ \\
Sondagem linear     & $\Theta(n)$ & $\Theta(n)$ \\
\hline
\end{tabular}
\end{table}
