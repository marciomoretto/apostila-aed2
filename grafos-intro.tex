\chapter{Introdução a Grafos}

Grafos são uma das abstrações mais versáteis da Computação. 
Sempre que precisamos modelar entidades e suas relações --- pessoas e amizades, cidades e estradas, páginas e links, tarefas e dependências --- estamos, essencialmente, lidando com grafos. 
Nesta aula, apresentaremos a ideia central, os conceitos básicos, os principais tipos, algumas propriedades elementares, o problema clássico das pontes de Königsberg, aplicações, e as representações mais usadas em código (matriz e lista de adjacência).

\section{O que são grafos}

Um \textbf{grafo} é formado por um conjunto de \textbf{vértices} (ou \emph{nós}) e um conjunto de \textbf{arestas} que conectam pares de vértices. 
Intuitivamente, vértices representam objetos e arestas representam relações entre esses objetos.

Exemplos cotidianos incluem:
\begin{itemize}
  \item \textbf{Rede social:} pessoas são vértices; amizades são arestas.
  \item \textbf{Mapa:} cidades são vértices; estradas são arestas.
  \item \textbf{Dependências:} tarefas são vértices; uma aresta indica que uma tarefa depende de outra.
\end{itemize}

\section{Conceitos fundamentais}

\begin{itemize}
  \item \textbf{Vértice (nó):} entidade básica que queremos modelar.
  \item \textbf{Aresta (ligação):} relação entre dois vértices.
  \item \textbf{Grau:} número de arestas incidentes a um vértice (em dígrafos, distinguimos \emph{grau de entrada} e \emph{grau de saída}).
  \item \textbf{Caminho:} sequência de vértices conectados por arestas.
  \item \textbf{Ciclo:} caminho que começa e termina no mesmo vértice.
  \item \textbf{Conectividade:} dois vértices estão conectados se existe um caminho entre eles; um grafo é \emph{conexo} se qualquer vértice alcança qualquer outro.
  \item \textbf{Componente conexa:} subgrafo conexo maximal.
\end{itemize}

\section{Tipos de grafos}

Dependendo da natureza da relação, usamos variantes de grafos:

\begin{center}
\renewcommand{\arraystretch}{1.3}
\begin{tabular}{@{}p{2.8cm} p{6cm} p{5cm}@{}}
\toprule
\textbf{Tipo} & \textbf{Descrição} & \textbf{Exemplo prático} \\
\midrule
\textbf{Não direcionado} & Arestas sem orientação (relação bidirecional). & Amizade em redes sociais. \\
\textbf{Direcionado (dígrafo)} & Arestas orientadas ($u \to v$). & Seguidores no Twitter, fluxo de dados. \\
\textbf{Ponderado} & Arestas com pesos (custo, tempo, distância). & Rotas de transporte. \\
\textbf{Multigrafo} & Arestas paralelas entre o mesmo par de vértices. & Linhas distintas entre duas cidades. \\
\textbf{Com laços} & Aresta que liga um vértice a si próprio. & Dependência recursiva. \\
\bottomrule
\end{tabular}
\end{center}


\subsection*{Diagramas ilustrativos}

% Estilos locais para os diagramas
\newcommand{\tikzstyleset}{
\tikzset{
  vertex/.style={draw,circle,minimum size=8mm,inner sep=0pt},
  edgeu/.style={-,thick},
  edged/.style={-{Latex},thick},
  lbl/.style={font=\small,fill=white,inner sep=1pt}
}
}

\paragraph{Grafo não direcionado.}
\begin{figure}[h]
\centering
\begin{tikzpicture}[node distance=22mm]
\tikzstyleset
  \node[vertex] (A) {A};
  \node[vertex, right=of A] (B) {B};
  \node[vertex, below=of $(A)!0.5!(B)$] (C) {C};
  \draw[edgeu] (A)--(B);
  \draw[edgeu] (B)--(C);
  \draw[edgeu] (C)--(A);
\end{tikzpicture}
\caption{Grafo não direcionado: arestas sem orientação.}
\end{figure}

\paragraph{Grafo direcionado (dígrafo).}
\begin{figure}[h]
\centering
\begin{tikzpicture}[node distance=24mm]
\tikzstyleset
  \node[vertex] (A) {A};
  \node[vertex, right=of A] (B) {B};
  \node[vertex, below=of $(A)!0.5!(B)$] (C) {C};
  \draw[edged] (A) -- (B);
  \draw[edged] (B) -- (C);
  \draw[edged] (C) -- (A);
\end{tikzpicture}
\caption{Grafo direcionado: arestas com sentido ($u\to v$).}
\end{figure}

\paragraph{Grafo ponderado.}
\begin{figure}[h]
\centering
\begin{tikzpicture}[node distance=28mm]
\tikzstyleset
  \node[vertex] (A) {A};
  \node[vertex, right=of A] (B) {B};
  \node[vertex, below=of $(A)!0.5!(B)$] (C) {C};
  \draw[edgeu] (A) -- (B) node[midway,above,lbl]{10};
  \draw[edgeu] (A) -- (C) node[midway,left,lbl]{5};
  \draw[edgeu] (B) -- (C) node[midway,right,lbl]{7};
\end{tikzpicture}
\caption{Grafo ponderado: cada aresta carrega um custo/distância/tempo.}
\end{figure}

\paragraph{Multigrafo (arestas paralelas).}
\begin{figure}[h]
\centering
\begin{tikzpicture}[node distance=40mm]
\tikzstyleset
  \node[vertex] (A) {A};
  \node[vertex, right=of A] (B) {B};
  \draw[edgeu, bend left=20]  (A) to node[above,lbl]{L1} (B);
  \draw[edgeu, bend right=20] (A) to node[below,lbl]{L2} (B);
\end{tikzpicture}
\caption{Multigrafo: mais de uma aresta entre o mesmo par de vértices.}
\end{figure}

\paragraph{Grafo com laço.}
\begin{figure}[h]
\centering
\begin{tikzpicture}[node distance=28mm]
\tikzstyleset
  \node[vertex] (D) {D};
  \node[vertex, right=of D] (E) {E};
  \draw[edgeu] (D) -- (E);
  \path (D) edge[loop above,thick] node[lbl]{w} ();
\end{tikzpicture}
\caption{Grafo com laço: aresta que sai e retorna ao mesmo vértice.}
\end{figure}

\section{Propriedades básicas}

\begin{itemize}
  \item \textbf{Ordem} de um grafo: número de vértices ($|V|$).
  \item \textbf{Tamanho} de um grafo: número de arestas ($|E|$).
  \item \textbf{Soma dos graus (não direcionado):} $\sum_{v\in V}\deg(v)=2|E|$.
  \item \textbf{Em dígrafos:} $\sum \deg^+(v)=\sum \deg^-(v)=|E|$.
  \item \textbf{Classes comuns:} grafos completos, bipartidos, regulares, árvores (conexos e acíclicos).
\end{itemize}

Essas relações ajudam a checar consistência e servem de base para algoritmos (por exemplo, muitos percorrem vizinhos de um vértice, logo custos dependem do grau).

\section{O problema das pontes de Königsberg}

Em 1736, Euler modelou o problema de atravessar todas as pontes da cidade de Königsberg sem repetir nenhuma. 
Abstraindo regiões como vértices e pontes como arestas, a pergunta torna-se: \emph{existe um caminho que use cada aresta exatamente uma vez?} (caminho euleriano).

O grafo abstraído possui \textbf{quatro vértices} com \textbf{graus ímpares}, o que impossibilita um percurso euleriano. 
O critério clássico diz que um grafo conexo tem:
\begin{itemize}
  \item \emph{circuito euleriano} se e somente se todos os vértices têm grau par;
  \item \emph{caminho euleriano} (não circuito) se e somente se exatamente dois vértices têm grau ímpar.
\end{itemize}

\paragraph{Abstração de Königsberg.}
\begin{figure}[h]
\centering
\begin{tikzpicture}[node distance=24mm]
\tikzstyleset
  \node[vertex] (A) {A};
  \node[vertex, right=of A] (B) {B};
  \node[vertex, below left=14mm and 10mm of A] (C) {C};
  \node[vertex, below right=14mm and 10mm of B] (D) {D};

  % conexões representativas (graus ímpares)
  \draw[edgeu] (A)--(B);
  \draw[edgeu] (A)--(C);
  \draw[edgeu] (A)--(D);
  \draw[edgeu] (B)--(C);
  \draw[edgeu] (B)--(D);
  \draw[edgeu] (C)--(D);
\end{tikzpicture}
\caption{Königsberg (abstração): todos os quatro vértices com grau ímpar $\Rightarrow$ sem caminho euleriano.}
\end{figure}

\section{Aplicações}

Grafos aparecem em:
\begin{itemize}
  \item \textbf{Redes sociais e informação:} comunidade, influência, recomendações.
  \item \textbf{Rotas e logística:} caminhos mínimos, cobertura, árvores geradoras.
  \item \textbf{Dependências e planejamento:} ordenação topológica em DAGs.
  \item \textbf{Computação e redes:} roteamento, conectividade, tolerância a falhas.
  \item \textbf{Ciências naturais:} interações biológicas, redes metabólicas, ecologia.
\end{itemize}

\section{Representações em código}

Para implementar algoritmos, precisamos escolher uma representação. As duas mais comuns são:

\subsection*{Matriz de adjacência}

Uma matriz $|V|\times|V|$ onde a entrada $(i,j)$ indica se há aresta entre $i$ e $j$ (ou o peso, em grafos ponderados).

\begin{itemize}
  \item \textbf{Espaço:} $O(|V|^2)$.
  \item \textbf{Teste de aresta $(u,v)$:} $O(1)$.
  \item \textbf{Iterar vizinhos de $u$:} $O(|V|)$.
  \item \textbf{Melhor para grafos densos.}
\end{itemize}

\subsection*{Lista de adjacência}

Para cada vértice, mantemos a lista dos seus vizinhos (com pesos, se houver).

\begin{itemize}
  \item \textbf{Espaço:} $O(|V|+|E|)$.
  \item \textbf{Teste de aresta $(u,v)$:} proporcional ao grau de $u$.
  \item \textbf{Iterar vizinhos de $u$:} $O(\deg(u))$.
  \item \textbf{Melhor para grafos esparsos.}
\end{itemize}

\subsection*{Exemplo concreto}

Considere o grafo não direcionado com $V=\{A,B,C,D\}$ e $E=\{\{A,B\}, \{A,C\}, \{B,C\}, \{C,D\}\}$.

\paragraph{Matriz de adjacência ($A\!=\!0, B\!=\!1, C\!=\!2, D\!=\!3$):}
\[
M =
\begin{bmatrix}
0 & 1 & 1 & 0 \\
1 & 0 & 1 & 0 \\
1 & 1 & 0 & 1 \\
0 & 0 & 1 & 0 \\
\end{bmatrix}
\]

\paragraph{Listas de adjacência:}
\[
\begin{aligned}
A &: \{B,C\} \\
B &: \{A,C\} \\
C &: \{A,B,D\} \\
D &: \{C\} \\
\end{aligned}
\]

\subsection*{Comparativo resumido}
\begin{center}
\renewcommand{\arraystretch}{1.2}
\begin{tabular}{@{}l c c@{}}
\toprule
 & \textbf{Matriz} & \textbf{Lista} \\
\midrule
Espaço & $O(|V|^2)$ & $O(|V|+|E|)$ \\
Testar $(u,v)$ & $O(1)$ & $O(\deg(u))$ \\
Iterar vizinhos & $O(|V|)$ & $O(\deg(u))$ \\
Esparso vs denso & pior & melhor \\
\bottomrule
\end{tabular}
\end{center}

\section{Encaminhamento}

Nas próximas aulas, usaremos essas representações para implementar \textbf{buscas} (BFS e DFS), \textbf{caminhos mínimos} (Dijkstra, Bellman--Ford) e \textbf{árvores geradoras mínimas} (Kruskal, Prim). 
A escolha entre matriz e lista impacta tempo e memória, portanto vale fixar bem essas diferenças desde já.

\medskip
\noindent\textbf{Exercícios sugeridos.}
\begin{enumerate}
  \item Dado um grafo pequeno, escreva a matriz e as listas de adjacência.
  \item Calcule os graus e identifique componentes conexas.
  \item Classifique: direcionado? ponderado? multigrafo? há laços?
  \item No grafo de Königsberg, verifique os graus e conclua sobre a existência de caminho/circuito euleriano.
\end{enumerate}
