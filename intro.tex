\chapter{Introdução}

Estruturas de dados e algoritmos são os pilares fundamentais da ciência da computação. 
Enquanto algoritmos são sequências finitas de instruções bem definidas para resolver um problema ou realizar uma tarefa, estruturas de dados são formas de organizar, armazenar e acessar informações de maneira eficiente. 
Juntos, eles permitem que possamos resolver problemas complexos de forma sistemática, com clareza, precisão e eficiência. 
O mesmo problema pode ser abordado por diferentes algoritmos, e a escolha da estrutura de dados adequada muitas vezes determina o desempenho final da solução.

Compreender estruturas de dados e algoritmos é essencial para escrever programas corretos, legíveis e eficientes. 
Saber quando usar uma tabela de dispersão em vez de uma lista, ou uma árvore balanceada em vez de um vetor ordenado, é o que distingue uma solução ingênua de uma solução robusta e escalável. 
Além disso, a análise de algoritmos -- tanto no pior caso quanto em média -- fornece ferramentas para comparar alternativas e justificar decisões de projeto. 
Estudar esses temas é, portanto, um exercício de precisão lógica, mas também de criatividade na construção de soluções.

Estas notas foram elaboradas para acompanhar a disciplina e pressupõem que os alunos já tenham familiaridade com os fundamentos da análise de algoritmos -- como notação assintótica, complexidade de tempo e espaço, e noções básicas de recursão --, conteúdos geralmente abordados na disciplina {\em Introdução à Análise de Algoritmos}. 
Também se espera conhecimento prévio sobre estruturas de dados elementares, como vetores, listas ligadas, pilhas e filas, que são introduzidas em {\em Algoritmos e Estruturas de Dados 1}.

Nesta disciplina, o foco será o estudo de três classes centrais de estruturas de dados: tabelas de dispersão (hashes), árvores de busca e grafos. 
Esses temas serão tratados do ponto de vista conceitual e prático, com ênfase na análise de desempenho e nas aplicações típicas de cada estrutura. 
O objetivo é proporcionar uma base sólida que permita aos alunos projetar, implementar e analisar algoritmos eficientes para problemas que envolvem organização e busca de informação.

Nesta apostila, as estruturas de dados serão apresentadas inicialmente de forma abstrata, por meio de especificações em termos de Tipos Abstratos de Dados (TADs), seguidas por implementações em pseudocódigo e em linguagem C. 
A descrição abstrata permite compreender a funcionalidade esperada de cada estrutura sem se comprometer com detalhes de implementação, promovendo uma visão mais clara e geral. 
O pseudocódigo, por sua vez, serve como ponte entre a abstração conceitual e a codificação concreta, permitindo expressar os algoritmos de maneira legível e independente de linguagem. 
As implementações em C buscam fornecer uma visão prática e eficiente, permitindo ao aluno experimentar diretamente com os conceitos em exercícios e projetos.

Nesta apostila, os conteúdos são organizados em doze capítulos, acompanhando a sequência didática da disciplina. 
Iniciamos com uma introdução ao conceito de Tipos Abstratos de Dados (TADs), com ênfase no TAD dicionário e em suas duas variações principais: o dicionário simples e o dicionário ordenado. 
Em seguida, tratamos das tabelas de dispersão e funções de hash, abordando o conceito de função de dispersão, o fator de carga e aplicações como verificação de duplicatas e contagem de frequência.

Os capítulos seguintes se dedicam a técnicas de tratamento de colisões em tabelas de hash, incluindo encadeamento e sondagem linear, com menções a estratégias mais avançadas como sondagem quadrática e duplo hash. 
Também discutimos o desempenho dessas estruturas, analisando casos médios e piores, e revisitamos listas ligadas como suporte para algumas implementações.

Na segunda parte do curso, o foco se volta para árvores. 
Iniciamos pelas árvores binárias de busca, estudando sua implementação e uso como dicionários ordenados. 
Em seguida, apresentamos três variantes de árvores balanceadas: 
as árvores AVL, as árvores B (adequadas para acesso em disco) e as árvores vermelho-preto, que realizam balanceamento automático por meio de marcações nos nós.

A última parte da apostila é dedicada aos grafos. 
Apresentamos os conceitos fundamentais e representações possíveis, com exemplos de aplicações reais. 
Estudamos algoritmos de busca em grafos, como busca em largura (BFS) e busca em profundidade (DFS), e algoritmos para cálculo de caminhos mínimos, incluindo Dijkstra e Bellman-Ford. 
Encerramos com os algoritmos de Kruskal e Prim para obtenção de árvores geradoras mínimas, fundamentados no teorema do corte.

\section*{Objetivos}

O principal objetivo desta disciplina é aprofundar o conhecimento dos alunos sobre estruturas de dados e algoritmos, com foco na escolha e implementação de representações eficientes para problemas computacionais complexos. 
Pretende-se desenvolver a capacidade de raciocínio algorítmico, aliando clareza conceitual à análise crítica de desempenho.

Especificamente, ao final da disciplina, espera-se que o aluno seja capaz de:
\begin{itemize}
  \item Compreender e implementar estruturas de dados fundamentais como tabelas de dispersão, árvores de busca e grafos;
  \item Avaliar a eficiência de algoritmos em diferentes contextos, utilizando ferramentas de análise assintótica;
  \item Reconhecer padrões de problemas e selecionar estruturas e algoritmos adequados para resolvê-los;
  \item Aplicar os conceitos estudados em implementações práticas, com ênfase na correção, clareza e desempenho do código.
\end{itemize}

\section*{Bibliografia}

\begin{itemize}
  \item Goodrich \& Tamassia. {\em Estruturas de Dados e Algoritmos em Java} (Bookman, 2007).
  \item Sedgewick \& Wayne. {\em Algorithms, 4th Edition}.
\end{itemize}

