
\documentclass[12pt,oneside]{article}

% ---------- Packages ----------
\usepackage[utf8]{inputenc}
\usepackage[T1]{fontenc}
\usepackage[brazilian]{babel}
\usepackage{amsmath,amssymb}
\usepackage{amsthm}
\usepackage{enumitem}
\usepackage{geometry}
\usepackage{hyperref}
\usepackage{lmodern}

\geometry{letterpaper,margin=2.5cm}

% ---------- Environments ----------
\newtheorem{exercicio}{Exercício}

% ---------- Metadata ----------
\title{Prova: Tabelas de Dispersão (Hashing)}
\author{}
\date{\today}

\begin{document}
\maketitle


\begin{exercicio}
Considere uma tabela de dispersão com $m = 17$ posições e $n = 20$ elementos.
\begin{enumerate}[label=\alph*)]
  \item Calcule o fator de carga $\alpha$. Supondo uso de encadeamento, qual o comprimento médio das listas?
  \item Se adotássemos endereçamento aberto com sondagem linear, o que aconteceria? 
\end{enumerate}
\end{exercicio}

\begin{exercicio}
Analise a importância da escolha da função de hash.
\begin{enumerate}[label=\alph*)]
  \item Explique por que uma função de hash deve ser eficiente tanto em tempo de cálculo quanto em distribuição dos valores.
\end{enumerate}
\end{exercicio}

\begin{exercicio}
Compare métodos de sondagem em endereçamento aberto.
\begin{enumerate}[label=\alph*)]
  \item Descreva a diferença entre sondagem linear, sondagem quadrática e hashing duplo.
  \item Quais desses métodos tendem a reduzir o problema de \emph{clustering primário}? Explique.
\end{enumerate}
\end{exercicio}



% ---------- (Opcional) Gabarito Resumido ----------
%\chapter*{Gabarito Resumido}
%\begin{enumerate}
%  \item \textbf{Exercício 1:} (a) $\alpha = n/m = 20/17 \approx 1{,}176$; (b) comprimento médio $\approx \alpha$; (c) $\alpha>1$ inviável em endereçamento aberto, alto custo esperado $\Rightarrow$ rehashing.
%  \item \textbf{Exercício 2:} Eficiência computacional + distribuição uniforme; exemplos: multiplicação (Knuth), misturas bitwise; uniformidade evita piores casos.
%  \item \textbf{Exercício 3:} Linear: $h(k,i)=h(k)+i$; Quadrática: $h(k,i)=h(k)+c_1 i + c_2 i^2$; Duplo: $h(k,i)=h_1(k)+i\cdot h_2(k)$; duplo reduz clustering primário.
%\end{enumerate}

\end{document}