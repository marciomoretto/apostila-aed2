\documentclass[12pt,oneside]{article}

% ---------- Packages ----------
\usepackage[utf8]{inputenc}
\usepackage[T1]{fontenc}
\usepackage[brazilian]{babel}
\usepackage{amsmath,amssymb}
\usepackage{amsthm}
\usepackage{enumitem}
\usepackage{geometry}
\usepackage{hyperref}
\usepackage{lmodern}

\geometry{letterpaper,margin=2.5cm}

% ---------- Environments ----------
\newtheorem{exercicio}{Exercício}

% ---------- Metadata ----------
\title{Gabarito: Tabelas de Dispersão (Hashing)}
\author{}
\date{\today}

\begin{document}
\maketitle

\begin{exercicio}
\begin{enumerate}[label=\alph*)]
  \item $\displaystyle \alpha=\tfrac{n}{m}=\tfrac{20}{17}\approx 1{,}18.$  
  Em encadeamento, esse valor corresponde ao comprimento médio das listas.
  \item Em endereçamento aberto, $\alpha$ não pode ultrapassar $1$. Como $n>m$, não é possível inserir todos os elementos. Quando $\alpha$ se aproxima de $1$, o custo das operações cresce muito devido ao \emph{clustering primário}, exigindo rehashing com uma tabela maior.
\end{enumerate}
\end{exercicio}

\begin{exercicio}
A função de hash precisa ser rápida de calcular e distribuir as chaves de maneira uniforme, para minimizar colisões e manter as operações em tempo médio constante.  
Exemplo: a função multiplicativa $h(k)=\lfloor m\,(kA-\lfloor kA\rfloor)\rfloor$, com $A$ irracional, garante boa dispersão para inteiros.
\end{exercicio}

\begin{exercicio}
\begin{enumerate}[label=\alph*)]
  \item \textbf{Linear}: procura posições consecutivas.  
  \textbf{Quadrática}: aumenta a distância entre sondagens de forma quadrática.  
  \textbf{Duplo hashing}: usa uma segunda função de hash para definir o passo.
  \item Quadrática e duplo hashing reduzem \emph{clustering primário}; o duplo hashing ainda evita \emph{clustering secundário}, pois passos variam entre as chaves.
\end{enumerate}
\end{exercicio}


\end{document}
