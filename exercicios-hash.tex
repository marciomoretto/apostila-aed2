\chapter{Exercícios}

\begin{exercicio}
Considere uma tabela de dispersão com $m = 13$ posições e $n = 9$ elementos.
\begin{enumerate}
  \item Calcule o fator de carga $\alpha$.
  \item Supondo uso de encadeamento, qual é o comprimento médio das listas?
  \item O que aconteceria com $\alpha$ e com o desempenho das operações se inserirmos mais 10 elementos sem aumentar o tamanho da tabela?
\end{enumerate}
\end{exercicio}

\begin{exercicio}
Explique o papel da função de hash no desempenho das tabelas de dispersão.
\begin{enumerate}
  \item O que significa dizer que uma função de hash é “uniforme”?
  \item Dê um exemplo simples de função de hash não uniforme e explique por que ela pode causar degradação de desempenho.
  \item Qual a relação entre a qualidade da função de hash e o pior caso de busca?
\end{enumerate}
\end{exercicio}

\begin{exercicio}
Compare as estratégias de resolução de colisão.
\begin{enumerate}
  \item Descreva a diferença entre encadeamento e endereçamento aberto com sondagem linear.
  \item Por que o encadeamento é considerado mais robusto em cenários de alto fator de carga?
  \item O que é \emph{clustering primário} e como ele afeta a sondagem linear?
\end{enumerate}
\end{exercicio}

\begin{exercicio}
Considere um sistema com sondagem linear e fator de carga $\alpha = 0{,}75$.
\begin{enumerate}
  \item Estime o custo médio de uma busca bem-sucedida e uma mal-sucedida com base nas fórmulas de Knuth (não é necessário justificar a fórmula).
  \item E se $\alpha = 0{,}95$?
  \item Por que o controle de $\alpha$ é mais crítico em sondagem linear do que em encadeamento?
\end{enumerate}
\end{exercicio}

\begin{exercicio}
Analise os seguintes cenários de desempenho:
\begin{enumerate}
  \item Qual é o custo médio da busca bem-sucedida em encadeamento com $\alpha = 3$?
  \item Qual é o pior caso para busca usando qualquer método de dispersão?
  \item Explique como o uso de rehashing pode garantir que o custo médio de busca permaneça constante.
\end{enumerate}
\end{exercicio}

