\documentclass[12pt,oneside]{article}

% ---------- Packages ----------
\usepackage[utf8]{inputenc}
\usepackage[T1]{fontenc}
\usepackage[brazilian]{babel}
\usepackage{amsmath,amssymb}
\usepackage{amsthm}
\usepackage{enumitem}
\usepackage{geometry}
\usepackage{hyperref}
\usepackage{lmodern}

\geometry{letterpaper,margin=2.5cm}

% ---------- Environments ----------
\newtheorem{exercicio}{Exercício}

% ---------- Metadata ----------
\title{Lista de Exercícios: Árvores}
\author{}
\date{\today}

\begin{document}
\maketitle

\begin{exercicio}
Simule passo a passo a construção de uma árvore binária de busca inserindo, nessa ordem, as chaves:
\[ 40,\, 20,\, 60,\, 10,\, 30,\, 50,\, 70. \]
\begin{enumerate}[label=\alph*)]
  \item Desenhe a árvore resultante após todas as inserções.
  \item Em seguida, remova a chave $20$ e redesenhe a árvore.
  \item Explique qual caso de remoção ocorreu (nó folha, nó com um filho ou nó com dois filhos).
\end{enumerate}
\end{exercicio}

\begin{exercicio}
Usando a árvore construída no exercício anterior:
\begin{enumerate}[label=\alph*)]
  \item Escreva a sequência das chaves em \textbf{pré-ordem}, \textbf{em ordem} e \textbf{pós-ordem}.
  \item Explique brevemente o significado de cada tipo de percurso e para que tipo de operação cada um é mais adequado.
\end{enumerate}
\end{exercicio}

\begin{exercicio}
\begin{enumerate}[label=\alph*)]
  \item Explique o que significa uma árvore estar \textbf{balanceada}.
  \item Por que o balanceamento é importante para o desempenho das operações de busca?
  \item Dê um exemplo de árvore com sete nós que não está balanceada e explique como o tempo de busca piora em relação à árvore balanceada com o mesmo número de nós.
\end{enumerate}
\end{exercicio}

\begin{exercicio}
Considere a sequência de inserções em uma árvore AVL:
\[ 10,\, 20,\, 30,\, 40,\, 50,\, 25. \]
\begin{enumerate}[label=\alph*)]
  \item Mostre passo a passo como a árvore se transforma a cada inserção.
  \item Indique onde ocorrem rotações e qual o tipo de rotação (simples à esquerda, simples à direita, dupla à esquerda ou dupla à direita).
  \item Explique o efeito de cada rotação sobre o fator de balanceamento dos nós envolvidos.
\end{enumerate}
\end{exercicio}

\begin{exercicio}
Compare conceitualmente as árvores AVL e as árvores vermelho-preto.
\begin{enumerate}[label=\alph*)]
  \item Compare as estratégias de \textbf{balanceamento} das duas estruturas. Em que sentido uma é mais rigorosamente balanceada que a outra?
  \item Liste as \textbf{cinco propriedades fundamentais} das árvores vermelho-preto.
  \item Explique por que as árvores vermelho-preto são consideradas ``aproximadamente balanceadas'' e como isso afeta o desempenho das operações em relação às AVL.
\end{enumerate}
\end{exercicio}

\end{document}
